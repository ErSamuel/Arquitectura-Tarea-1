\documentclass[12pt]{article}

\title{Resumen}
\author{Samuel Reyna}

\begin{document}
\maketitle

   \textbf{Registro y Palabras de memoria}\\
      Son unidades de almacenamiento. En MIPS, los registros almacenan datos para realizar operaciones aritméticas,
      mientras que las palabras de memoria almacenan estructuras de datos, tablas y registros desbordados. Otra característica 
      a destacar es que los registros son más fáciles de utilizar, ya que no necesitas una instrucción específica de transferencia 
      para acceder a los datos.
     
     \textbf{Instrucciones}\\
     MIPS contiene distintos tipos de instrucciones, aunque las principales
     las podemos categorizar de la siguiente manera:

     \begin{itemize}
        \item \textbf{aritméticas}: son operaciones de suma y resta, en donde se operan
         solo dos datos para guardar el resultado en un tercero.
        \item \textbf{Transferencia de datos} son instrucciones que permiten mover información
        entre registros y memoria. Se dividen en dos tipos de instrucciones, cargar y almacenar.
        \item \textbf{Lógicas}: realizan operaciones bit a bit sobre registros, los cuales nos
        permiten ejecutar operaciones booleanas, y operaciones de desplazamiento.
        \item \textbf{salto condicional}: permiten cambiar el flujo del programa solo si se cumple una condición específica.
        \item \textbf{salto incondicional}: similar al punto anterior, pueden cambiar el flujo del
        programa pero estas instrucciones no necesitan evaluar ninguna condición.
     \end{itemize}


\end{document}